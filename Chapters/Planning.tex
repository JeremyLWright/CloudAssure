\section{Planning Methodology}
Our data transfer decision framework is designed to use intelligent planning
techniques such as MDP (Markov Decision Process), POMDP (Partially Observable
Markov Decision Process) and NOMDP (Non Observable Markov Decision Process) to
make real time decisions on data transfers within an organizations cloud
network. The purpose of these techniques is to provide a mathematical framework
to model some decision making. Consider the scenario of a user (node) in an
organization wanting to send classified data to another user (node) in the cloud
environment. The sender has no idea if the recipient is trustworthy enough to
handle such critical files. The current practice of using static policies to
guide decision making is not sufficient as it is infeasible to enumerate all
possible policies for all possible scenarios. Moreover generating run time
decisions through policies is slow. Thus, there exists a need for a dynamic,
real-time decision process to address these problems. 

We address these issues using intelligent planning techniques. The planning
algorithms in our framework make decisions based on probabilistic values which
are estimated from the trust values of users/nodes in a stochastic environment
\autocite{JMarecki2012}, \autocite{JWu.2011}. The algorithms then take into consideration the classification level
of the document and the trust metric of the specific user(s) who are to receive
the classified documents and generate plans which will specify whether the
user/node is really trustworthy enough to handle such data.  

In the process of
developing the algorithm, we have explored three kinds of models: MDP, POMDP and
NOMDP. Initially the plan was to use a POMDP as the main decision maker in the
framework since it is able factor uncertainty into the decision making process.
Uncertainty arises from the fact that all the states (the model) of the cloud
environment is not always known or observable. However, after careful research
in this area we decided not to pursue using a POMDP because the POMDP complexity
is very high \autocite{LeslieP.Kaelbling1998}, \autocite{Zilberman}, 20]. 
That is, computationally it is very expensive and the
process requires a lot of time to generate decisions, which would eventually
slow down the system. The solution is to use MDP to make the decisions. In order
to use an MDP however, all uncertainty needs to be removed from the system.
Hence, we make the assumption that all the states and entirety of the model of
the cloud environment is known and observable. This is a valid assumption
because in an organization all nodes are known and the structure/model of nodes
is well defined. As a real world example this would be an organization’s IT
manager knowing all the workstations and network connections within the company.

Now let us explore the structure of the Markov Decision Process (MDP). It
has four tuples in total $< S, A, T, R >$ \autocite{QimingHe2000}:
\begin{description}
    \item{S (States):} Represents a finite set of states present in the cloud Environment.
    \item{A (Actions):} Represents a finite set of actions that are possible in
        every state $Sn$ of the cloud environment.
    \item{T (Transition Function):} Represents the probabilistic values of the occurrence of every action from every state. This can be derived from the known model/structure of the cloud environment.
    \item{R (Reward/ Cost function):} Represents Reward value at each state, which
    in our case is the trust value of every node/user.  
\end{description}
The MDP tuples above are
specifically modeled and/or changed to meet our requirements for our data
transfer decision framework. In order for the MDP to operate we require the
information specified by the above tuples, with the addition of a start
state and goal state. Once a user/node has made the request to transfer data
to another user/node, a Security Agent takes over. The security agent (SA)
contacts the SA at the recipient user/node level. The recipient SA next
gathers data on the nodes around it, and gets their trust values from the
centralized trust server. This data is used to initiate a localized graph
algorithm. This algorithm generates a model of the nodes neighboring the
recipient node, and is n-levels deep, where n is a certain order of node
distance from the recipient. Here n can be tweaked to get an optimal result.
After this localized graph is generated, the MDP algorithm is run on it with
the recipient as the start node and an imaginary malicious node with
relatively high trust value as the goal node. The idea here is to evaluate
the neighbors of the recipient, and find out if the MDP decides it is ok to
transmit the data to the malicious node or not. This can give rise to
a large number of situations such as:
\begin{enumerate}
    \item If the recipient’s trust value is low, but the trust values of all the
        neighbors are high then the risk of data being leaked is relatively low. The
        decision can be made to send the data.
    \item If the recipient’s trust value is high, but the trust values of all the
        neighbors are relatively moderate, then the risk of data being leaked is
        relatively low. The decision can be made to send the data.
    \item If the recipient’s trust value is low and the trust values of one of the
        neighbors is also low, then the risk of data being leaked is relatively
        high. The decision can be made not to send the data.
    \item If the recipient’s trust value is high but the trust values of the neighbors
are low, then the risk of data being leaked is relatively high. The decision
can be made not to send the data.  
\end{enumerate}
For any transmission that occurs, the
documents that are sent over the network are archived in a database of the
Security agents. This is done for archival purposes and also to check for
plagiarism within the organization’s network. Whenever a user/node creates
a document a diff function is performed with all the documents that the node
has been in contact with. This diff function scans the document and
calculates the similarity value. The similarity value is used as a factor in
the trust re-evaluation stage which in turn affects the decision making
process. As an example, whether a user leaks an email by forwarding it, or
copying the content into a new message and sending it, the system will catch
the leak and modify trust values accordingly.

\section{Results on Planning}
We decided to implement our planning system with the
simulated values because all the other modules namely classification and trust
re-evaluation were being developed simultaneously, hence we were not able to
achieve the exact workflow. We made sure that simulated values are consistent
with the proposed outputs of the classification and the trust re-evaluation
modules. After finalizing the framework and doing a preliminary analysis from
\autocite{Norvig2012}, we decided to focus on the Markov decision process to generate our
decisions.  One of the other issues we faced while practically implementing our
system is that whether to choose a finite or an infinite horizon, which is
whether an agent should change its policies according to the time is has or the
policies generated by it must be stationary. By studying our problem we felt it
was better to implement a stationary policy generation mechanism, which assumes
that the agent has infinite time to perform its operations. This is because if
we implement an non-stationary policy generation mechanism in our system one of
the uses might be that according to factors such as day of time, month, levels,
etc. the decision generated may change which inserts lot of inconsistencies.
Policies generation inconsistencies are very difficult to handle and needs more
effort and time to introduce more parameters to control the inconsistency and
make it consistent.  There are many techniques to generate the decision using
MDP \autocite{Wikipedia2013}, namely policy iteration, modified policy iteration, priority sweeping
and value iteration. To address our problem statement we have used policy and
value iterations because they are more efficient, more accurate and relevant to
the solution we expect to achieve. This is because for modified policy iteration
and priority sweeping to work effectively they need more accurate and stable
heuristics, which is very difficult to achieve given the dynamic nature of our
domain where the trust values are constantly re-evaluated and updated. 

\section{Value Iteration}
The value iteration mainly works by using a technique called dynamic
programming. The algorithm constantly does a backward induction using a bellman
equation \autocite{Wikipedia2013} and this technique is called Bellman Backup. In the case of the
value iteration the initial policy is not assumed and is policy is constantly
updated when the values are updated, here the values are nothing but the trust
values of each node.  The equation that is used in value iteration for infinite
horizon is,  


We used the above equation to calculate the intermediate results, that is the
reward value based on which the decision will be made. In the above equation the
V represents the value function and T represents the reward metric,    is the
discount factor mainly used as a control parameter. The M in the equation
represents the model (in our case it can be the structure of the network), that
is model specifies what an agent can do in each state/node and what the outcome
is likely to be, s represents the current state/node, a represents the actions,
represents the next state/node and k represents the number of states/nodes in
network. Then the function is iteratively called to constantly update the reward
metrics (intermediate result, not used/stored after decision is made).

We had implemented the algorithm using Matlab Script in the Matlab, and the
initial parameters that were set are that there is a goal state with a high
reward value this is done to see if any attack vectors can be possible because
all agents will try to reach it, a null state through which an agent may not be
able to pass through and this is done to simulate a impassable node, rest of the
states have a relatively equal value of -0.04 and this is done so that the agent
has no incentive to stay indefinitely in the state/node and accumulate the
reward metrics.
 



 
 
As we can see the values keep on changing on every iteration, updated by the
backward induction of the bellman equation continuously. Every reward metric in
each iteration is unique. So when does it reach termination, that depends on the
maximum allowed error for state values that we set in our model. 

When the program reaches the maximum allowed error value it terminates. This
does not mean that the policy changes, policy will stabilize after a few
iterations based on the discount values we set, but the values in the value
iteration function will keep on changing until it reaches the maximum allowed
error. We call this the epsilon factor. And value iteration may terminate early
if set the maximum allowed error value accordingly.  We can see that in figures
represented, the values keep changing continuously and corresponding action also
changes in accordance to the reward metrics. Since we do not start with the
initial set of policies the value function will keep on generating different
action sets that constitutes the policy. After few runs the action set will
stabilize and we might call it a policy. But it is important to let the
algorithm to continue to run even though the policy has stabilized, the reason
being that the algorithm may be stuck at a local minimum and may need some time
to get out of it.  That is why it is very important to carefully select the
discount factor.

 
 

Here from the above figure we can see that the action set has stabilized and we
have a policy to reach the goal state. In reference to our problem statement if
an agent reaches a goal state, that is  an simulated node/user who has an
artificially high trust values surrounded by nodes with very low trust value
then there is an attack possibility and the decision is made not to transfer the
documents.

\section{Policy Iteration}

The policy iteration works almost similar to value iteration but the difference
mainly being that in policy iteration we start the algorithm with some random
policy (set of actions) and then use the backward induction to improve the
policy. The algorithm constantly does a backward induction using a bellman
equation \autocite{Wikipedia2013} and this technique is called Bellman Backup. In the case of the
policy iteration the initial policy is assumed and is policy is constantly
evaluated and updated when the values are updated, here the values are nothing
but the trust values of each node.  The equation that is used in value iteration
for infinite horizon is,  




The different parameters used in the above equation are the same as the ones in
the value iteration, the difference being that   represents a policy. The brief
description on working of the algorithm is as follows, 
\begin{enumerate}
    \item Choose a random policy $\pi$
    \item Loop: 
        \begin{enumerate}
            \item Evaluate $V\pi$ 
            \item For each $s$ in $S$, set improved policy
            \item Replace $\pi$ with $\pi\prime$ 
        \end{enumerate}
        \item Until no improving action possible at any state
\end{enumerate}

The policy improvement is made using the below formula,



In the below figure we can see the several runs of the policy iteration
algorithms, here we start with the random policy and use the equations to update
the reward metrics of the every state and there updating the policy if any
better policy is found than one we started with or one that we have. We can see
the action sets stabilize at the end and hence arriving upon a policy that will
help us make the correct decision.
 
 
\section{Running time}

\begin{itemize}
    \item Clearly from the above example we can see that the Value Iteration takes
more resources and running time compared to others. This is because of the fact
that there is no policy in the beginning. So the value iteration frames policies
as it proceeds. And of course the running time of the value iterations largely
depend on the maximum error value that we set.
    \item The policy iteration takes much less running time when we compare it to the
value iteration in this case. This is because of the fact that we start with an
arbitrary random policy. This will be a huge improvement because its like giving
a model to the function even though it may be wrong. The value will be computed
according to the initial policy setting. If it is correct it is unchanged, if
not then we change the policy and re-compute the value again.
\end{itemize}
By running both the policy and value iteration we conclude the following thing,
\begin{itemize}
    \item We were able to make the informed decisions on data transfer decisions by
using good heuristics, which are nothing but simulated values for the data
classification and the trust values for the nodes/users
    \item The running time for
the algorithms is reasonable, we achieved convergence for a 12 state problem in
about (will give you the values, trying to improve it), which is reasonable. But
more modification needs to be done if we are to scale the system.
\end{itemize}
