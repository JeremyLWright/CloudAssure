\newacronym{DLP}{DLP}{Data Leak Protection}
%TODO Add the High Level System overview to this section.
\section{Background and Motivation}
A 2008 study commissioned by Cisco Systems of more than 2000 IT professionals in 10 countries revealed some startling statistics [1]. Of these professionals, 39% were more concerned about threats from their own employees, than from malicious attackers outside of the company. The reason for this concern? Almost half of the professionals questioned reported that in addition to employees dealing with more information than ever before, and at a quicker pace, they are also not receiving the proper education in data security awareness. What this all adds up to is a shocking statistic: 11% of the professionals reported that either they or an employee they knew of, accessed unauthorized information and sold it for profit. That is 1 out of every 10 employees. In a world where a source code file, design document, or other digital file worth millions can be leaked via email, USB drive, internet, etc., this creates a huge problem.
According to a May 2009 US federal government report, between 2008 and 2009 American business losses due to cyber-attacks have grown to more than $1 trillion worth of intellectual property[1]. The impact of unauthorized data transmission can also go beyond monetary damages. For example, the HBGary scandal resulted in the imprisonment of several formerly successfully security professionals [2]. Thus a comprehensive data transfer mechanism is required to mitigate the impact of unauthorized data transmission in cloud-based systems.
The problem of data leakage has yet to find a catch-all solution. Everyday millions of bits of data are transferred between users and computers in organizations both large and small [4]. Who is monitoring these transmissions? Obviously, from large sums of money, to intellectual property, to even jail time, the consequences of unauthorized data transmission are massive. Software such as firewalls, policy checkers, virus scanners and others attempt to help prevent this unauthorized transmission, but the software is often niche, focused on one specific task, and proprietary. What if there was a system that could be integrated into an organization, trained specifically to deal with a company�s data, and provide security over the entire network. This is where our framework: �CloudAssure� comes into the picture. 

Goals and scope of study of this project
The objective of this project is to develop a data transfer decision framework that makes informed decisions on data transmission among nodes (users) in a cloud-based system, a two part process is used; dynamic trust evaluation of nodes, and classification of data sensitivity. The goals that we plan to meet in our project in order to work toward achieving the aforementioned objective are as follows:
�	Development of an effective framework to address unauthorized data transmission through security agents which enact and enforce data transmission decisions in order to prevent and/or mitigate unauthorized data transmission in cloud-based systems.
�	Development of a planning model using a Markov Decision Process (MDP) to make a decision on whether to transmit data or not.
�	Development of a standardized classification algorithm to aid security agents running the MDP�s in making data transmission decisions among nodes by assessing the sensitivity of the data to be transferred in cloud environments. 
�	Development of a Trust Evaluation Equation to aid the planning methodology in making a transmission decision.
The scope of this project is the generation of a theoretical model to make decisions based on dynamic trust values obtained from continuous node evaluation and the sensitivity of data obtained from the data classification stage. If possible, we would like to develop a practical implementation of our proposed classification algorithm. The following are the assumptions we make:
�	We assume that the system periphery is adequately protected (no outsider attacks)
�	There is an organizational structure (hierarchy) within the workforce.
�	One time critical data leak prevention is not within our threat model.
�	We assume the user identity/system is not compromised by a malicious actor.

\section{Current State of the Art}
The three-factor approach that CloudAssure uses to prevent unauthorized data
transmission relies on its three elements (Classification, Trust Evaluation, and
Modeling) to all integrate seamlessly and function as a united service providing
a defense in depth strategy. As mentioned in the introduction, the problem that
CloudAssure hopes to solve is that modern solutions are often privatized,
expensive, niche, or domain specific. CloudAssure takes a general approach,
which once trained on a specific organization's data, can be integrated into any
organizational system.  

There is no doubt that the current state-of-the-art
solutions in managing unauthorized data transmission and preventing data loss
are in the private commercial sector. Heavy hitters such as Cisco systems \autocite{Cisco2008},
Symantec \autocite{Symantec2013}, and McAfee \autocite{McAfee2013} all provide a wide range of
software/hardware/consulting tools and services to provide the required security
to their customers. To an organization that can afford the cost, which is
relatively prohibitive to smaller businesses (especially since data security is
an often undervalued aspect), an enterprise solution provides a strong
defense-in-depth data security solution. However, this approach has benefits and
drawbacks.  

The biggest benefit of current enterprise solutions is of
course their comprehensiveness. Symantec's \gls{DLP} suite of
products includes: \gls{DLP} for Mobile, \gls{DLP} for Network, \gls{DLP} for endpoint (client and
server), \gls{DLP} for storage, and a \gls{DLP} Enforcement server. These solutions are all
proprietary, utilize dedicated hardware and software. Additionally, this
solutions only provides niche
security unless the entire product suite is purchased, including Symantec's
specialists to configure, monitor, train, and provide support \autocite{Symantec2013}.

Both McAfee and Cisco provide solutions very similar to that of Symantec. McAfee's
\gls{DLP} product suite includes a \gls{DLP} Manager, \gls{DLP} Discover,
\gls{DLP} Monitor, \gls{DLP}
Prevent all as an integrated hardware/software, and \gls{DLP} Endpoint and Device Control
software \autocite{McAfee2013}. Cisco's product list includes IronPort Email Security Appliance,
IronPort Web Security Appliance, Loss Prevention Services, Security Agent, and
IronPort Data Loss Prevention which again are either integrated proprietary
hardware/software solutions, or a direct software product \autocite{Cisco2008}. Both companies
encourage potential customers to purchase the full suite of products in addition
to requiring their specialists be involved in the process.  

Obviously the cost
of an enterprise solution must be taken into consideration. Enterprise solutions
have built-in costs of specialized hardware, system configuration and
maintenance considerations. Lastly, do to the integrated nature of these systems possible modifications to 
\emph{organizational infrastructure}, not to mention additional personnel costs.

\section{Open-Source Offerings}
On the other end of
the spectrum are open-source software and tools. Snort \autocite{Parker2013} and
Suricata \autocite{Jonkman2013}
are open source traffic monitoring solutions that can be used as an intrusion
detection system, and to classify network traffic. OpenDLP \autocite{Gavin2012} is an open
source tool designed to classify data. Academic software such as Cornell
University's Spider \autocite{Cornell2013} could also be used to identify confidential data in
files. Because this software is free and/or open source however, it brings with
it inherent risks.  

A research report done by Christopher Hoke for the SANS
institute attempted to build a comprehensive defense-in-depth solution for
managing \gls{DLP} \autocite{Hoke2012}.  Hoke's approach was to build and configure a virtual
machine that could essentially run on-top of an organization's network and
utilize a combined suite of open-source or free tools to emulate the defense
strategies used by the enterprise solutions. The results, while promising, still
leave much to be desired.  

Hoke notes himself however, that while his solution
works and is feasible, there are drawbacks. Hoke's solution was unable to handle
any form of encrypted data traffic. Also, his system was unable to monitor
application to host communication. Due to the nature of the system, this type of
solution also requires a non-insignificant time and knowledge investiture for
proper setup, utilization, and effectiveness.  

In between Enterprise and
Open-Source lie products such as MyDLP \autocite{MyDLP2013}. MyDLP is provided as an open
source, licensed all-in-one product. MyDLP is designed to run on a single
server, and provide modular security. Some included modules include MyDLP
Network for network monitoring, MyDLP Endpoint for inspecting user operations,
and MyDLP Web UI for managing the system. The obvious benefit of solutions such
as MyDLP is their extensibility to specific domains if the IT Personnel wish to
make their own modifications.  

The CloudAssure framework sits in between these
two approaches. It does not require proprietary hardware such as the enterprise
approaches, and the hardware services it does use and/or require can be easily
configured to run on existing infrastructure. Likewise, the setup of the three stages of
classification, trust evaluation, and modeling are accessible enough to be able
to be done in house by IT personnel and System Administrators. If the
CloudAssure framework is made open-sourced, then this provides the extensibility
and benefits (and risks) inherent in that model. 

