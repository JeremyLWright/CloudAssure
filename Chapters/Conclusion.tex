\section{Current State of the Art}
The three-factor approach that CloudAssure uses to prevent unauthorized data
transmission relies on its three elements (Classification, Trust Evaluation, and
Modeling) to all integrate seamlessly and function as a united service providing
a defense in depth strategy. As mentioned in the introduction, the problem that
CloudAssure hopes to solve is that modern solutions are often privatized,
expensive, niche, or domain specific. CloudAssure takes a general approach,
which once trained on a specific organization's data, can be integrated into any
organizational system.  

There is no doubt that the current state-of-the-art
solutions in managing unauthorized data transmission and preventing data loss
are in the private commercial sector. Heavy hitters such as Cisco systems \autocite{Cisco2008},
Symantec \autocite{Symantec2013}, and McAfee \autocite{McAfee2013} all provide a wide range of
software/hardware/consulting tools and services to provide the required security
to their customers. To an organization that can afford the cost, which is
relatively prohibitive to smaller businesses (especially since data security is
an often undervalued aspect), an enterprise solution provides a strong
defense-in-depth data security solution. However, this approach has benefits and
drawbacks.  

The biggest benefit of current enterprise solutions is of
course their comprehensiveness. Symantec's \gls{DLP} suite of
products includes: \gls{DLP} for Mobile, \gls{DLP} for Network, \gls{DLP} for endpoint (client and
server), \gls{DLP} for storage, and a \gls{DLP} Enforcement server. These solutions are all
proprietary, utilize dedicated hardware and software. Additionally, this
solutions only provides niche
security unless the entire product suite is purchased, including Symantec's
specialists to configure, monitor, train, and provide support \autocite{Symantec2013}.

Both McAfee and Cisco provide solutions very similar to that of Symantec. McAfee's
\gls{DLP} product suite includes a \gls{DLP} Manager, \gls{DLP} Discover,
\gls{DLP} Monitor, \gls{DLP}
Prevent all as an integrated hardware/software, and \gls{DLP} Endpoint and Device Control
software \autocite{McAfee2013}. Cisco's product list includes IronPort Email Security Appliance,
IronPort Web Security Appliance, Loss Prevention Services, Security Agent, and
IronPort Data Loss Prevention which again are either integrated proprietary
hardware/software solutions, or a direct software product \autocite{Cisco2008}. Both companies
encourage potential customers to purchase the full suite of products in addition
to requiring their specialists be involved in the process.  

Obviously the cost
of an enterprise solution must be taken into consideration. Enterprise solutions
have built-in costs of specialized hardware, system configuration and
maintenance considerations. Lastly, do to the integrated nature of these systems possible modifications to 
\emph{organizational infrastructure}, not to mention additional personnel costs.

\section{Open-Source Offerings}
On the other end of
the spectrum are open-source software and tools. Snort \autocite{Parker2013} and
Suricata \autocite{Jonkman2013}
are open source traffic monitoring solutions that can be used as an intrusion
detection system, and to classify network traffic. OpenDLP \autocite{Gavin2012} is an open
source tool designed to classify data. Academic software such as Cornell
University's Spider \autocite{Cornell2013} could also be used to identify confidential data in
files. Because this software is free and/or open source however, it brings with
it inherent risks.  

A research report done by Christopher Hoke for the SANS
institute attempted to build a comprehensive defense-in-depth solution for
managing \gls{DLP} \autocite{Hoke2012}.  Hoke's approach was to build and configure a virtual
machine that could essentially run on-top of an organization's network and
utilize a combined suite of open-source or free tools to emulate the defense
strategies used by the enterprise solutions. The results, while promising, still
leave much to be desired.  

Hoke notes himself however, that while his solution
works and is feasible, there are drawbacks. Hoke's solution was unable to handle
any form of encrypted data traffic. Also, his system was unable to monitor
application to host communication. Due to the nature of the system, this type of
solution also requires a non-insignificant time and knowledge investiture for
proper setup, utilization, and effectiveness.  

In between Enterprise and
Open-Source lie products such as MyDLP \autocite{MyDLP2013}. MyDLP is provided as an open
source, licensed all-in-one product. MyDLP is designed to run on a single
server, and provide modular security. Some included modules include MyDLP
Network for network monitoring, MyDLP Endpoint for inspecting user operations,
and MyDLP Web UI for managing the system. The obvious benefit of solutions such
as MyDLP is their extensibility to specific domains if the IT Personnel wish to
make their own modifications.  

It follows that there is a need for such a data management framework. 
The CloudAssure framework attempts to leverage the benefits of these approaches
while minimizing vendor lock-in. It does not require proprietary hardware such as the enterprise
approaches, and the hardware services it does use and/or require can be easily
configured to run on existing infrastructure. Likewise, the setup of the three stages of
classification, trust evaluation, and modeling are accessible enough to be able
to be done in house by IT personnel and System Administrators. If the
CloudAssure framework is made open-sourced, then this provides the extensibility
and benefits (and risks) inherent in that model. 

\section{Comparison}
\begin{enumerate}
    \item Advantages (Claims)
        \begin{enumerate}
            \item Makes informed decisions based on well derived heuristics          	
            \item Not depend on static policies to make decisions
            \item Heuristics are derived dynamically through trust reevaluation and classification
            \item Easy enforcement
            \item Easy adaptation to any organizational environment
            \item Doesn�t require proprietary or predator vendor lockin.
            \item 
        \end{enumerate}
    \item Disadvantages
        \begin{enumerate}
            \item Might be computationally expensive
            \item Might be difficult to scale up beyond certain limit
            \item Might not work if organization does not have a hierarchical set up
            \item System needs to be trained by expert before it becomes efficient
            \item Might not work if we assume system compromise or ``trusted
                employee'' attack.
        \end{enumerate}
\end{enumerate}

\section{Recommendations}
\begin{enumerate}
    \item Classification
        \begin{enumerate}
            \item 	Weakness � requires an expert and a great deal of time to train the system.
            \item 	Weakness � requires a large repository of classified data to implement the system.
            \item 	Weakness � Provided algorithms may not be sufficient to classify the needs of a specific organization.
        \end{enumerate}
    \item 	Trust Evaluation
        \begin{enumerate}
            \item 	Strength � Deterministic Trust value
            \item 	Weakness � Trust evaluation is tailored to our specific system
            \item 	Frequent trust evaluation is computationally expensive
            \item 	Recommendation � Reduce the cost of reevaluating the trust metric
        \end{enumerate}
\end{enumerate}

\section{Future Work}
\begin{enumerate}
\item Trust Evaluation
    \begin{enumerate}
        \item Document Similarity. The \( \%Doc \) parameter in the trust equation comes from a yet implemented module.
        \item Informed Randomized Scheduling of Trust Reevaluation to reduce the computation cost.
        \item Example: Reevaluating at a low server load period
        \item Split algorithm into a computationally inexpensive value for real-time calculations, and a heavy duty computation computed at a scheduled point. 
    \end{enumerate}
\item Planning
    \begin{enumerate}
        \item Use POMDP because current planning algorithm is not tolerant of uncertainty. 
        \item Try to convert stochastic actions to deterministic actions and do FF-Replan, which can improve performance in decision making.
    \end{enumerate}
    \item Classification
        \begin{enumerate}
            \item Deeper analysis of Neural Nets for classification as it provides a non-linear solution.
        \end{enumerate}
\end{enumerate}

