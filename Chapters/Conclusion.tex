\section{Current State of the Art}
The three-factor approach that CloudAssure uses to prevent unauthorized data
transmission relies on its three components (Classification, Trust Evaluation, and
Modeling) to integrate seamlessly and function as a united service, providing
a defense in depth strategy. As mentioned in the Section~\ref{sec:introduction},
CloudAssure hopes to solve the problem that modern solutions are often privatized,
expensive, niche, or domain specific. CloudAssure takes a general approach,
which once trained on a specific organization's data, can be integrated into any
organizational infrastructure.  

The private commercial sector, currently provides the current state-of-the-art
solutions in managing unauthorized data transmission and preventing data loss. 
Heavy hitters such as Cisco systems \autocite{Cisco2008},
Symantec \autocite{Symantec2013}, and McAfee \autocite{McAfee2013} all provide a wide range of
software/hardware/consulting tools and services to provide the required security
to their customers. To an organization that can afford the cost, which is
relatively prohibitive to smaller businesses (especially since data security is
an often undervalued aspect), an enterprise solution provides a strong
defense-in-depth data security solution. However, this approach has benefits and
drawbacks.  

The biggest benefit of current enterprise solutions is of
course their comprehensiveness. Symantec's \gls{DLP} suite of
products includes: \gls{DLP} for Mobile, \gls{DLP} for Network, \gls{DLP} for endpoint (client and
server), \gls{DLP} for storage, and a \gls{DLP} Enforcement server. These solutions are all
proprietary, utilize dedicated hardware and software. Additionally, these
solutions only provide niche or domain specific
security unless the entire product suite is purchased, including hiring Symantec's
specialists to configure, monitor, train, and provide support \autocite{Symantec2013}.

Both McAfee and Cisco provide solutions very similar to that of Symantec. McAfee's
\gls{DLP} product suite includes a \gls{DLP} Manager, \gls{DLP} Discover,
\gls{DLP} Monitor, \gls{DLP}
Prevent all as an integrated hardware/software, and \gls{DLP} Endpoint and Device Control
software \autocite{McAfee2013}. Cisco's product list includes IronPort Email Security Appliance,
IronPort Web Security Appliance, Loss Prevention Services, Security Agent, and
IronPort Data Loss Prevention which again are either integrated proprietary
hardware/software solutions, or a direct software product \autocite{Cisco2008}. Both companies
encourage potential customers to purchase the full suite of products in addition
to requiring their specialists be involved in the process.  

Obviously the cost
of an enterprise solution must be taken into consideration. Enterprise solutions
have built-in costs of specialized hardware, system configuration and
maintenance considerations. Lastly, due to the integrated nature of these
systems organizations are left to adjust their work-flows, rather than the tools
responding to the organization's needs. 

\section{Open-Source Offerings}
On the other end of
the spectrum are open-source software and tools. Snort \autocite{Parker2013} and
Suricata \autocite{Jonkman2013}
are open source traffic monitoring solutions that can be used as an intrusion
detection system, and to classify network traffic. OpenDLP \autocite{Gavin2012} is an open
source tool designed to classify data. Academic software such as Cornell
University's Spider \autocite{Cornell2013} could also be used to identify confidential data in
files. However since, this software is free and/or open source, it brings with
it inherent risks.  

A research report done by Christopher Hoke for the SANS
institute attempted to build a comprehensive defense-in-depth solution for
managing \gls{DLP} \autocite{Hoke2012}.  Hoke's approach was to build and configure a virtual
machine that could essentially run on-top of an organization's network and
utilize a combined suite of open-source or free tools to emulate the defense
strategies used by the enterprise solutions. The results, while promising, still
leave much to be desired.  

Hoke notes himself however, that while his solution
works and is feasible, there are drawbacks. Hoke's solution was unable to handle
any form of encrypted data traffic. Also, his system was unable to monitor
application to host communication \autocite{Hoke2012}. Due to the nature of the system, this type of
solution also requires a non-insignificant time and knowledge investiture for
proper setup, utilization, and effectiveness.  

In between Enterprise and
Open-Source lie products such as MyDLP \autocite{MyDLP2013}. MyDLP is provided as an open
source, licensed all-in-one product. MyDLP is designed to run on a single
server, and provide modular security. Some included modules include MyDLP
Network for network monitoring, MyDLP Endpoint for inspecting user operations,
and MyDLP Web UI for managing the system. The obvious benefit of solutions such
as MyDLP is their extensibility to specific domains if the IT Personnel wish to
make their own modifications.  

It follows that there is a need for such a data management framework. 
The CloudAssure framework attempts to leverage the benefits of these approaches
while minimizing vendor lock-in. It does not require proprietary hardware such as the enterprise
approaches, and the hardware services it does use and/or require can be easily
configured to run on existing infrastructure. Likewise, the setup of the three stages of
classification, trust evaluation, and modeling are accessible enough to be able
to be done in house by IT personnel and System Administrators. If the
CloudAssure framework is made open-sourced, then this provides the extensibility
and benefits (and risks) inherent in that model. 

\section{System and Module Conclusions}
\subsection{CloudAssure Conclusions}
CloudAssure strikes a clear distinction between the current open-source
offerings, and their commercial counterparts. Yet, implementation aside, it is
evident that companies with any type of critical intellectual property can
benefit from a systematic approach to data leakage. Either through CloudAssure,
or some other offering, data leakage is a vastly expensive risk. Furthermore,
with the volume of data increasing exponentially, employees cannot be expected to
properly secure all data at all times. There is an increasingly critical
requirement within organizations to systematically control their data.

From designing, refining, and testing the CloudAssure framework, we have been able
to conclude that there exists a need for this type of product. Small businesses,
individuals, and even large organizations wishing to save some money could utilize
CloudAssure within their networks, with minimal headache. The unique aspects of
informed decision making and feedback learning make CloudAssure more than just a
policy enforcement tool. An organization that tailors CloudAssure to their data,
configures the system properly, and executes the system properly will be receiving
protection from unauthorized data transmission both in the present, and in the future.

\subsection{Classification Conclusions}
Classification techniques vary among their underlying models, predictive goals,
but one fact remains that classification, as much of machine learning, is
a process reversal. The reasons to why one would need to do this vary according
to application, with respect to CloudAssure our goal is clear, ``protect important
information''. As terse a thesis as this, the implementation is difficult. What is
important? How important? Is important a binary property or are there varying
levels? Does all data need to be protected or just some of it? There questions
cannot all be resolved algorithmically, rather some are more human questions than
pragmatic ones; However among these, there are questions we can address.

CloudAssure comes with two built in classification algorithms. The Classifier is
build against a Python class interface, thus one could extend CloudAssure with
their own implementation, however our results are based on simply these. The
Decision Tree, for our data set, required more memory, and more time to build
a classification model, while the Bayesian was
must cheaper, computationally. However, while both classify a document
quickly, the decision tree is 4\% more accurate. Within our specific
implementation the Decision Tree is the classifier to use. This leads to the
point that algorithms must be ``explored'' for maximum effectiveness. 

There exist very powerful tools such as Orange which permit exploratory data
analysis. As one explores a dataset with the intent of classifying, or extracting some
other underlying pattern the dataset needs to be transformed, and visualization is
the key interface between our strong human skills, and the computer's strong
numerical skills. Exploratory data analysis tools allow one to display data in
a multitude of formats. One can group, and ungroup data according to some
property. The computer complies with each of the numerical requests, while we
the data designer watches the data transform. Furthermore, there is a great deal
of research in how to visualize data to maximize our ability to cull
information from it. This is the purpose of machine learning and by extension
classification: to advance the human intellect through data. Data which
increases in quality by the minute. As a race we generate more data today than
we ever have before. The data is useless unless we can extract and learn from
it. Much like how history is the best teacher, we have cataloged, indexed and
searchable history. More accurate, and more available than ever before, yet with
the sheer volume comes its own challenges, and data analysis researchers are
leading the charge to this lofty challenge. 

\subsubsection{Recommendations}
There are numerous classification techniques, algorithms, feature kernels, and
powerful tools to help evaluate their effectiveness on a particular
organization's data. The greatest recommendation within
classification is to study your data, and evaluate a number of feature
techniques to determine the greatest value for your implementation. 

With respect to implementation, Python is a very strong tool. Its
performance is enhanced with high performance libraries, such and NumPy
and SciPy. Furthermore, many open-source packages such as NLTK provide intuitive
interfaces to these extremely high performance, FORTRAN optimized libraries.
Python is a strong candidate for implementation of CloudAssure Modules. After
the initial test phases, if an organization finds Python's performance lacking,
python permits a low cost (computational cost), interface to C. Critical modules
can be ported to C to increase performance several fold. 

These recommendations however demonstrate CloudAssures weaknesses within the
classification modules. Exploratory data analysis is required to extract the
best features from the data, but this analysis requires a machine learning
expert to do well. This hurdle must be overcome before CloudAssure can achieve
mass adoption. Lastly, to begin, an organization must already have a large
repository of classified data. If an organization is new, it may not have this
yet, or if the training data is insufficiently small, CloudAssure will offer
little protection. 


\subsection{Planning Conclusions}
We decided to implement our planning system with 
simulated values because the other modules of CloudAssure (Classification and Trust
Evaluation) were being developed simultaneously. Because of this we were not able to
achieve the exact workflow of a total system implementation. We made sure that simulated values are consistent
with the proposed outputs of the classification and the trust evaluation
modules. After finalizing the framework and doing a preliminary analysis from
\autocite{Norvig2012}, we decided to focus on the \gls{mdp} to generate our
decisions.  

Another issue, we faced while developing CloudAssure, is whether to choose
a finite or an infinite horizon: which is
whether an agent should change its policies according to the time is has or
should the
policies generated be stationary. In this sense, a policy is the rules the
\gls{mdp} will use to make its data transmission decisions.
After studying our problem, we felt it was better to implement a non-stationary policy generation mechanism, which assumes
that the agent has infinite time to perform its operations. This non-stationary policy generation mechanisms may need to consider factors such as time, day, month, organizational levels,
etc. Because of these factors, the decision generated may change which could introduce inconsistencies in decisions.
Inconsistent policy generation is very difficult to handle and requires more
effort and time to introduce more parameters to control the inconsistency and
make it dependable.  

Within the \gls{mdp} itself, there are many techniques to generate decisions \autocite{Wikipedia2013}, namely policy iteration, modified policy iteration, priority sweeping
and value iteration. To address our problem statement we have used policy and
value iterations because they are more efficient, more accurate and relevant to
the solution we expect to achieve. Conversely, for modified policy iteration
and priority sweeping to work effectively they need more accurate and stable
heuristics, which is very difficult to achieve given the dynamic nature of our
domain where the trust values are constantly evaluated and updated. 

\subsubsection{Planning Recommendations}
here

\subsection{Trust Evaluation Conclusions}
Trust is a powerful technique to ensure that the transmission of data among
organizations� employees is safe and secure. In this project, the transmission
of the data among the nodes is monitored by security agents. The input of trust
model are based on \emph{classification} of the data, the \emph{Similarity} (\(
\%Doc \))of the data meaning how much percentage of the data has been copied or
used in another document, the \emph{old trust value} (\(Tr_{old}\)). Every time the node interacts
with its neighbors, its old trust value is updated and the total number of
documents shared by a particular node at a particular instance. CloudAssure has
introduced and analyzed a simple, yet robust \emph{trust value equation} that support
decisions in the other phases of the framework. The output is a value between
0 and 1. We also reported a set of initial experimental results, demonstrating
the feasibility, effectiveness, and benefits of our trust equation .The output
result we be sent to the planning phase.

\subsubsection{Advantages (Claims)}
The Trust value equation in the Trust Management system is very useful and
ensures the transmission of data among the nodes. The High secured data is
maintained and avoids data leakage to the node which has a very low trust value
or a node which is not trusted. The Security agents play a vital role by
monitoring the data flow between the nodes of same level. Becoming aware of the
trust value, the node will be very cautious before sending the data to the other
node. All the document details are maintained separately for the trust value
calculation thus preserving the data for future use.

\subsubsection{Disadvantages}
Though the trust value equation is very helpful in making decisions, the
computation of trust value is expensive. This is confined to a specific model
having the similar setup as above. If the organization has different setup then
this equation will not work. This might not work if we assume system compromise
or ``trusted employee'' attack happens.

\subsubsection{Recommendations}
We need to strengthen the trust value for the system to be efficient. As our
trust evaluation is confined to our specific system, a general trust equation
has to be developed so that all the organizations can use and prevent the data
leakage. Also, in our process we calculate the trust values very frequently
which will lead to costly process. We need to come up with a new computation
process to minimize the cost of reevaluating the trust metric.

\section{Approach Comparison}
\begin{enumerate}
    \item Advantages of CloudAssure compared to other solutions
        \begin{enumerate}
            \item Makes informed decisions based on well derived heuristics.
            \item Does not depend on static policies to make decisions.
            \item Heuristics are derived dynamically through trust reevaluation and classification.
            \item Easy enforcement.
            \item Easy adaptation to any organizational environment.
            \item Doesn't require proprietary products or predatory vendor lockin.
            \item Framework is modular, and configurable.
        \end{enumerate}
    \item Disadvantages of CloudAssure compared to other solutions
        \begin{enumerate}
            \item Some operations may be computationally expensive.
            \item Framework may be difficult to scale up beyond certain theoretical limit.
            \item Framework may not work if an organization is not heirarchical.
            \item The System needs to be trained before it becomes efficient.
            \item System compromizations and ``trusted
                employee'' attacks were outside the scope of CloudAssure.
        \end{enumerate}
\end{enumerate}

\section{Future Work}
\subsection{Entire System}
Due to scheduling constraints each of the subcomponents were implemented
simultaneously. However, only the implementation of classification was initially
proposed. While internal consistency was fiercely managed, each component
currently exists, somewhat in isolation. The Classification components are
implemented in Python, while the planning algorithms are implemented in Matlab.
Furthermore, the classification component uses a single files database (SQLite),
which will not meed the high concurrency needs of the trust server, or the
request rates of a \gls{sa}. With the proof of concepts defined here, an overhaul
with respect to performance, and future maintenance needs to be performed.

\subsection{Planning}
Our future work is to figure out how to use POMDP because current planning
algorithm is not tolerant of uncertainty. Dealing with uncertainty allows us to
broaden the scope of our framework and let us apply this framework for real
world problems. Other major way we can reduce the complexity of the problem is
to try to convert stochastic actions to deterministic actions and do FF-Replan,
which can improve performance in decision making because deterministic decision
making is much more cost effective.  

\subsection{Classification} 
Within the classification modules of CloudAssure, we only implemented linear
techniques.  This is a marked gap in our survey. Non-linear techniques such as
Neural nets,
offer an extremely advanced classification properties dues directly to their
non-linearity. We choose to skip these in order to tune the linear techniques,
but the linear techniques reviewed are insufficient to my strong claims about
the generality of CloudAssure's classification effectively. 

\subsection{Trust Evaluation}
In the future, the Document similarity i.e. the \( \%Doc \) parameter in the
trust equation has to be implemented separately. The module will determine the
amount of data of a document that has been used in another document. For
reducing the computation cost of trust value calculation, an Informed randomized
scheduling process needs to be done. The scheduling shall include the
re-evaluating the trust values at a low server load period. Also, A Split
algorithm into a computationally inexpensive value for real-time calculations
and a heavy duty computation computed at a scheduled point.  
